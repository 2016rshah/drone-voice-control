\documentclass{article}

\usepackage[doublespacing]{setspace}

\usepackage[margin=1in]{geometry}

\usepackage{epigraph}

\usepackage{listings}
\usepackage{coloremoji}
\usepackage{hyperref}

\title{Voice Command Interface for Parrot A.R Drone}
\author{Rushi Shah}

\begin{document}

	\begin{singlespace}
		\maketitle
		\setlength{\epigraphwidth}{0.35\textwidth}
		\epigraph{``See no evil, hear no evil, speak no evil.''}{🙈🙉🙊}
	\end{singlespace}

	% \begin{abstract}
	% Program a drone to respond to voice commands.
	% \end{abstract}

	\section{Problem/Purpose/Engineering Goals}
	I aim to provide a voice command platform for the Parrot A.R Drone. These voice commands will be issued to either a web application on a desktop or through a mobile app by the user. The voice commands will be processed through a back-end written in Node.js, and may perhaps include computations such as computer vision or object tracking. The translated commands will then be sent to the drone to be executed.

	% \section{Background}
	% Here is the text of your Background.

	\section{Research Techniques/Methods}
	With the help of my project, users will be able to issue voice commands to either a web-app or a mobile-app to control a drone. In the case of a web-app, the user interface will include a video stream of from the drone in order to facilitate remote control. 

	\section{Materials}

	\subsection{Hardware Required}
	\begin{description}
	  \item[Parrot A.R Drone] \hfill \\
	  Drone to be controlled through voice commands. 
	  \item[Mobile Phone or Desktop/Laptop] \hfill \\
	  The input vector for the user to issue their voice commands. 
	\end{description}

	\subsection{Software Required}
	\begin{description}
	  \item[NodeJS] \hfill \\
	  The backend will be written in NodeJS. The backend will be responsible for recieving voice commands, processing them (as necessary), and transmitting the resulting command to the drone. 
	  \item[Drone API] \hfill \\
	  \url{https://github.com/felixge/node-ar-drone}

	  This NodeJS API wrapper will be used to issue the translated commands to the drone. 

	  \item[Mobile Phone Technology] \hfill \\
	  If the voice commands are issued through an app, then the app must be programmed with Android (Java). 

	  \item[Web Application Technology] \hfill \\
	  If the voice commands are issued through a web app, then assorted web technologies (like HTML, and Javascript) will be utilized 
	\end{description}

	% \section{References}
	% Here is the text of your References.

\end{document}